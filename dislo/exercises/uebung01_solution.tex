\documentclass{disloloesung}

\usepackage{enumerate}

\name{
    Marc Küper 2028220
}
\semester{WS20/21}
\tutor{Bob Niemand}
\uebung{01}
\class{Diskrete Strukturen und Logik}

\begin{document}

\begin{sheet}{1}

    \begin{aufgabe}{1.1}{}
	\begin{enumerate}[(a)]
	  \item $|K| = 26; |G| = 26, |Z| = 10;$ also: 26+26+10=62 Möglichkeiten; \\ Für 8 Stellen gibt es, wenn Wiederholung erlaubt ist, $62^8$ Möglichkeiten
	  \item $26*26*10*62^5$
	\end{enumerate}
    \end{aufgabe}

    \begin{aufgabe}{1.2}{}
	Die Anzahl der Mengen ist die Summe aller $a_{i}$ plus 1 (Nullmenge)
	$(\sum\limits_{i=1}^{n} a_{i}) +1$ \\
	So enthält jede Menge jeweils 1 Element einer Menge $S_{i}$
    \end{aufgabe}

    \begin{aufgabe}{1.3}{}
	Beweis durch Widerspruch:\\
	Sagen wir die obrige Aussage ist wahr, dann wähle ich 4 Ecken, sodass jede Würfelkante genau einmal verwendet wird.\\ Über die Gaußsche Summenformel 
	weiß ich, dass $\sum\limits_{i=1}^{12} i = 78$\\
	Als nächstes teile ich 78 durch 4 und erhalte ich die $magische$ Zahl 19.5, dies ist die Eckensumme.\\ Da aber die Eckensummen gleich und ganzzahlig sein müssen, ist hier ein Widerspruch und damit muss die Annahme, dass die obrige Aussage wahr ist, falsch sein.
    \end{aufgabe}
    
    \begin{aufgabe}{1.4}{}
	lol
    \end{aufgabe}

\end{sheet}

\end{document}
